\documentclass[8pt]{article}
\usepackage[spanish]{babel}
\usepackage[utf8]{inputenc}
\usepackage{amsmath}
\usepackage{amssymb}
\usepackage{amsthm}
\usepackage{amsfonts}
\usepackage{graphicx}
\usepackage{float}
\usepackage{subfig}
\usepackage{caption}
\usepackage{hyperref}
\usepackage{color}
\usepackage{pgfplots}
\usepackage{listings}
\usepackage{multicol}
\usepackage{blindtext}
\usepackage{parskip}
\usepackage{boxedminipage}
\usepackage{minipage-marginpar}
\usepackage{tikz}
\usepackage{fancybox}
\usepackage{enumitem}
\usepackage{cancel}
\usepackage{multirow}
% \usepackage{mathscr}
% \usepackage{}
% \usepackage{tikz-cd}
% \usepackage{tikz-qtree}
% \usepackage{tikz-qtree-compat}
% \usetikzlibrary{arrows,
%                 calc, 
%                 decorations.pathmorphing, 
%                 through}
% \usetikzlibrary{arrows.meta,bending,positioning}
\pgfplotsset{compat=1.16}

% Define a new counter:
\newcounter{definition}
\newcounter{teorema}
\newcounter{corolario}

\setcounter{definition}{0}
\setcounter{teorema}{0}
\setcounter{corolario}{0}

% Conjuntos:
\newcommand{\R}{\mathbb{R}}
\newcommand{\N}{\mathbb{N}}
\newcommand{\C}{\mathbb{C}}
\newcommand{\M}{\mathbb{M}}
\newcommand{\Z}{\mathbb{Z}}
\newcommand{\Q}{\mathbb{Q}}
\newcommand{\F}{\mathbb{F}}
\newcommand{\V}{\mathbb{V}}
\newcommand{\PP}{\mathbb{P}}
\newcommand{\HH}{\mathbb{H}}

% To create a fancy notes block with a new command:
% \newcommand{\note}[1]{\textcolor{red}{\textbf{Nota:} #1}}
% \newcommand{\note}[1]{
%     \begin{center}
%         \fbox{\begin{minipage}[b][1.5\height]%
%             [c]{0.8\textwidth} \textbf{Nota:} #1
%         \end{minipage}}
%     \end{center}
%     }
\newenvironment{teorema}[2][]{
    \addtocounter{teorema}{1}
    \begin{center}
        {\fboxsep=0.4cm \shadowbox{\begin{minipage}[b][1\height]%
            [c]{0.8\textwidth}
            \begin{center}
                \text{\textbf{Teorema \arabic{teorema}. \underline{#1}}}
            \end{center} \vspace*{0.2cm}
            #2
        \end{minipage}}}
    \end{center}
}

\newenvironment{corolario}[2][]{
    \begin{center}
        {\fboxsep=0.4cm \shadowbox{\begin{minipage}[b][1\height]%
            [c]{0.8\textwidth}
            \begin{center}
                \text{\textbf{Corolario \underline{#1}}}
            \end{center} \vspace*{0.2cm}
            #2
        \end{minipage}}}
    \end{center}
}

\newenvironment{definition}[2][Def]{
    \addtocounter{definition}{1}
    \begin{center}
        {\fboxsep=0.4cm \shadowbox{\begin{minipage}[b][1\height]%
            [c]{0.8\textwidth}
            \begin{center}
                \text{\textbf{Definición \arabic{definition}. \underline{#1}:}}
            \end{center} \vspace*{0.2cm}
            #2
        \end{minipage}}}
    \end{center}
}

\newenvironment{note}[1][...]{         
    \begin{center}
        {\fboxsep=0.3cm \shadowbox{\begin{minipage}[b][1\height]%
            [c]{0.8\textwidth}
            \begin{center}
                \textbf{Nota}
            \end{center} \vspace*{0.1cm}
            #1
        \end{minipage}}}
    \end{center}
}





% Column gap
\setlength{\columnsep}{1cm}

% Beautiful links style
\hypersetup{
    colorlinks=true,
    % linkcolor=blue,
    % filecolor=magenta,      
    % urlcolor=blue,
    % linkbordercolor={0 0 1},
}

% Beautiful page style
\usepackage[a4paper, margin=2.5cm, left=3.5cm, right=3.5cm]{geometry}
\usepackage{fancyhdr}
\pagestyle{fancy}
\fancyhf{}
\renewcommand{\headrulewidth}{0pt}
\renewcommand{\footrulewidth}{0pt}
\lhead{\textbf{Apuntes de Álgebra Lineal}}
\rhead{\textbf{Jorge Antonio Gómez García}}
\lfoot{\textbf{CIDE - Matemáticas III}}
\rfoot{\textbf{Otoño de 2022}}


\title{\textbf{Apuntes de Álgebra Lineal}}
\author{\textbf{Jorge Antonio Gómez García} \\ \textit{El conocimiento se construye en comunidad} \\ CIDE - Matemáticas III}
\date{Otoño de 2022}

% ####################################################################

\begin{document}
\maketitle

\begin{abstract}
    Este documento contiene los apuntes de Álgebra Lineal para la asignatura de Matemáticas III del Centro de Investigación y Docencia Económicas (CIDE) impartido por el profesor Itza Tlaloc Quetzalcoatl Curiel Cabral durante otoño del 2022. Puede ver el programa en \href{https://cideo365-my.sharepoint.com/:w:/g/personal/antonio_gomez_alumnos_cide_edu/EQY9B-QCIeBJpjJtvONTktkBMRqEuZp0hfoJHeW0zbszkw?e=SFeY7r}{este enlace}.
\end{abstract}

\tableofcontents

% ####################################################################

\section*{Contenido}

\begin{multicols}{2}
    \begin{enumerate}
        \item Matrices y eliminación gaussiana.
        \begin{enumerate}
            \item La geometría de las ecuaciones lineales.
            \item Eliminación gaussiana.
            \item Notación y operaciones con matrices.
            \item Factores triangulares.
            \item Inversas y transpuestas.
        \end{enumerate}
        \item Espacios vectoriales y ecuaciones lineales.
        \begin{enumerate}
            \item Espacios vectoriales.
            \item Subespacios vectoriales.
            \item Soluciones a $m$ ecuaciones lineales en $n$ incógnitas.
            \item Combinación lineal y espacio generado.
            \item Independencia y dependencia lineal.
            \item Base de un espacio y dimensión.
            \item Los cuatro espacios fundamentales.
            \item Transformaciones lineales.
            \item Matriz de una transformación lineal.
            \item Núcleo e imagen de una transformación lineal.
            \item Rango y nulidad.
        \end{enumerate}
        \item Ortogonalidad.
        \begin{enumerate}
            \item Producto interno.
            \item Conjuntos ortogonales y proyecciones  
            \item Aproximación de mínimos cuadrados. 
            \item Bases ortogonales, matrices ortogonales. 
            \item Procedimiento Gram-Schmidt.
        \end{enumerate}
        \item Determinantes.
        \begin{enumerate}
            \item Propiedades de los determinantes. 
            \item Fórmula del determinante.
            \item Aplicaciones.
        \end{enumerate}
        \item Eigenvalores y eigenvectores.
        \begin{enumerate}
            \item Cálculo de valores, vectores y espacios propios.
            \item La forma diagonal de una matriz.
            \item Ecuaciones en diferencia y las potencias $A^k$.
            \item Ecuaciones diferenciales y la exponencial $e^{At}$.
            \item Matrices complejas: Simétrica VS Hermitiana y Ortogonal VS Unitaria.
        \end{enumerate}
        \item Matrices positivas definidas.
        \begin{enumerate}
            \item Criterio de mínimos, máximos y puntos sillas de funciones de varias variables usando valores propios.
            \item Pruebas para determinar si es positiva definida.
            \item Matrices semidefinidas e indefinidas $Ax=\lambda Mx$.
        \end{enumerate}
    \end{enumerate}
\end{multicols}

% ####################################################################


\section*{Preámbulo}
\label{sec:preambulo}

\subsection*{Tipos de ecuaciones lineales}
\label{sec:0_tipos_de_ecuaciones_lineales}
\setcounter{equation}{0}
\setcounter{definition}{1}

Sean $A, B, C, m, x$ números reales y $x, y$ variables. Las ecuaciones lineales son aquellas que pueden expresarse de las siguientes formas:

\begin{enumerate}
\item Pendiente ordenada al origen de la recta:
\begin{align*}
    y = mx + b
\end{align*}
\item Forma normal de la recta:
\begin{align*}
    Ax + By + C = 0
\end{align*}
\item Forma general de la recta:
\begin{align*}
    Ax + By = C
\end{align*}
\item Forma punto pendiente de la recta:
\begin{align*}
    (y - y_1) = m(x - x_1)
\end{align*}
\item Forma simétrica de la recta:
\begin{align*}
    \frac{x}{a} + \frac{y}{b} = 1
\end{align*}
\end{enumerate}

\begin{note}
[
    Los resultados de investigación suelen entregarse en la forma general de la recta.
]
\end{note}

El álgebra lineal pretende conocer la información de los sistemas de ecuaciones lineales. Le interesa si un problema tiene una, muchas o ninguna solución.

\subsection*{Métodos de solución de ecuaciones lineales}
\label{sec:0_metodos_de_solucion_de_ecuaciones_lineales}

\begin{enumerate}
    \item Método de sustitución.
    \item Sumas y restas ($+/-$).
    \item Método de igualación.
    \item Método de Cramer.
    \item Método de Gauss-Jordan.
\end{enumerate}

\subsection*{Ejercicios}
\label{sec:0_ejercicios}

Resuelve los siguientes sistemas de ecuaciones lineales para $x$ y $y$:

\begin{enumerate}
    \item $x-y = 7 \\ x+y = 5$
    \item $x-y = 7 \\ 2x-2y = 14$
    \item $x-y = 7 \\ 2x-2y = 13$
\end{enumerate}

\subsection*{Notas sobre cada ejercicio}
\label{sec:0_notas_sobre_cada_ejercicio}

\begin{enumerate}
    \item Es un sistema \textbf{no singular}. \\ Tiene solución única. \\ Es resultado de la igualdad $n = n, \quad n \in \mathbb{R}$.
    \item Es un sistema \textbf{singular}. \\ Tiene infinitas soluciones. \\ Es resultado de la igualdad $0=0$. \\ La solución $\in (-\infty, \infty)$.
    \item Es un sistema \textbf{singular}. \\ No tiene solución. \\ Es resultado de la igualdad $0 = n, \quad n \in \mathbb{R}, \quad n \neq 0$. \\ Es decir, el resultado es una inconsistencia.
\end{enumerate}

\subsection*{Operadores gaussianos}
\label{sec:0_operadores_gaussianos}

\begin{enumerate}
    \item \textbf{Operador de suma de filas} (Suma y resta): \\ $R_{i} \leftarrow R_{i} + R_{j}$.
    \item \textbf{Operador de multiplicación por un escalar:} \\ $R_{i} \leftarrow \alpha R_{i}$.
    \item \textbf{Suma y resta + multiplicación por un escalar:} \\ $R_{i} \leftarrow \alpha R_{i} + R_{j}$.
    \item \textbf{Operador de intercambio de filas:} \\ $R_{i} \leftrightarrow R_{j}$.
\end{enumerate}

\subsection*{Notas sobre las matrices resultantes}
\label{sec:0_notas_sobre_las_matrices_resultantes}

Sea $A$ una matriz de $m \times n$ y sean $a, b, c, d, e, f$ números reales:

\begin{enumerate}
    \item \textbf{Determinante igual a cero}: $|A| = 0$ \\ El sistema tiene múltiples soluciones. \\ Ejemplo: $\left[ \begin{array}{rr|r} a & b & c \\ 0 & 0 & 0 \end{array} \right]$
    \item \textbf{Determinante igual a cero}: $|A| = 0$ \\ El sistema no tiene soluciones, es decir, es inconsistente. \\ Ejemplo: $\left[ \begin{array}{rr|r} a & b & c \\ 0 & 0 & f \end{array} \right], \quad f \neq 0$.
    \item \textbf{Determinante distinto de cero}: $|A| \neq 0$ \\ El sistema tiene una única solución. \\ Ejemplo: $\left[ \begin{array}{rr|r} a & b & c \\ d & e & f \end{array} \right]$
\end{enumerate}

\subsection*{Inversión de matrices}
\label{sec:0_inversion_de_matrices}

\subsubsection*{Inversión por cofactores}
\label{sec:0_inversion_por_cofactores}

El método de inversión de matrices por cofactores es una técnica para encontrar la inversa de una matriz cuadrada. Este método se basa en el concepto de cofactor de un elemento de una matriz y utiliza la fórmula de la inversa de una matriz dada por:

$$A^{-1} = \frac{1}{\det(A)}\text{adj}(A)$$

donde $\det(A)$ es el determinante de la matriz $A$ y $\text{adj}(A)$ es la matriz adjunta de $A$. La matriz adjunta de $A$ se puede calcular como:

$$\text{adj}(A) = C^T$$

donde $C$ es la matriz de cofactores de $A$. La matriz de cofactores de $A$ se define como:

$$C_{ij} = (-1)^{i+j} M_{ij}$$

donde $M_{ij}$ es el menor de $A$ que se obtiene eliminando la fila $i$ y la columna $j$ de la matriz $A$.

Para calcular la inversa de una matriz $A$ usando este método, primero se calcula el determinante de la matriz $A$ y luego se calcula la matriz de cofactores $C$ de $A$. Luego se calcula la matriz adjunta de $A$ como $C^T$. Finalmente, se utiliza la fórmula de la inversa de la matriz para calcular la inversa de $A$ como:

$$A^{-1} = \frac{1}{\det(A)}C^T$$

Este método funciona siempre y cuando el determinante de la matriz $A$ sea distinto de cero. Si el determinante de la matriz es cero, entonces la matriz no tiene inversa.

\subsubsection*{Inversión por operadores gaussianos}
\label{sec:0_inversion_por_operadores_gaussianos}

Este método se basa en el uso de operadores matriciales especiales llamados operadores gaussianos, que son matrices que tienen la propiedad de "cancelar" una fila o columna de una matriz al multiplicarla por ellos.

Para invertir una matriz $A$ usando este método, se sigue el siguiente proceso:

Sea $A$ una matriz cuadrada. Por ejemplo: \begin{align*}
    A = \left[ \begin{array}{rrr} a_{11} & a_{12} & a_{13} \\ a_{21} & a_{22} & a_{23} \\ a_{31} & a_{32} & a_{33} \end{array} \right]
\end{align*}

\begin{enumerate}
    \item Se obtiene la matriz identidad $I$ del mismo tamaño que $A$ mediante la concatenación de las matrices $A$ y $I$: $[A|I]$. Por ejemplo: \\ \begin{align*}
        \left[ \begin{array}{rrr|rrr} a_{11} & a_{12} & a_{13} & 1 & 0 & 0 \\ a_{21} & a_{22} & a_{23} & 0 & 1 & 0 \\ a_{31} & a_{32} & a_{33} & 0 & 0 & 1 \end{array} \right]
    \end{align*}
    \item Se aplican operadores gaussianos a la matriz $[A|I]$ para reducirla a una matriz escalonada reducida. Esto implica eliminar los elementos debajo de la diagonal principal mediante la multiplicación por operadores gaussianos adecuados. Por ejemplo: \\ \begin{align*}
        \left[ \begin{array}{rrr|rrr} b_{11} & b_{12} & b_{13} & c_{11} & c_{12} & c_{13} \\ 0 & b_{22} & b_{23} & c_{21} & c_{22} & c_{23} \\ 0 & 0 & b_{33} & c_{31} & c_{32} & c_{33} \end{array} \right]
    \end{align*}
    \item Una vez que la matriz se ha reducido a una matriz escalonada reducida, se aplican operadores gaussianos adecuados a la matriz para convertirla en la matriz identidad. Los operadores gaussianos se aplican a las filas superiores de la matriz para eliminar los elementos encima de la diagonal principal. Por ejemplo: \\ \begin{align*}
        \left[ \begin{array}{rrr|rrr} 1 & 0 & 0 & d_{11} & d_{12} & d_{13} \\ 0 & 1 & 0 & d_{21} & d_{22} & d_{23} \\ 0 & 0 & 1 & d_{31} & d_{32} & d_{33} \end{array} \right]
    \end{align*}
    \item La inversa de la matriz $A$ se encuentra en la parte derecha de la matriz resultante, es decir, en la sección que correspondía a la matriz identidad original. \\ \begin{align*}
        A^{-1} = \left[ \begin{array}{rrr} d_{11} & d_{12} & d_{13} \\ d_{21} & d_{22} & d_{23} \\ d_{31} & d_{32} & d_{33} \end{array} \right]
    \end{align*}
\end{enumerate}

Este método funciona siempre y cuando la matriz $A$ sea invertible, es decir, tenga un determinante distinto de cero. Si el determinante de la matriz es cero, entonces la matriz no tiene inversa y este método no puede utilizarse.
\pagebreak

% ####################################################################

\section{Matrices y eliminación gaussiana}
\label{sec:matrices_y_eliminacion_gaussiana}

\subsection{La geometría de las ecuaciones lineales}
\label{sec:1_la_geometria_de_las_ecuaciones_lineales}
\setcounter{equation}{0}
\setcounter{definition}{0}

Considere el siguiente sistema de ecuaciones lineales:\begin{align}
    \begin{cases}
        2x - y &= 1 \\
        x + y &=5
    \end{cases}
    \label{eq:1.1.1}
\end{align}

Este sistema de ecuaciones luce de la siguiente manera:

\begin{figure}[H]
    \centering
    \begin{tikzpicture}
        % Axes:
        \draw[line width=0.4, color=black, arrows=->] (-1, 0) -- (6, 0)
                                                      (0, -2) -- (0, 6);
        
        % Labels:
        \node[anchor=west] at (6, 0) {$x$};
        \node[anchor=south] at (0, 6) {$y$};

        % Grid:
        \foreach \x in {-1, 0, 1, 2, 3, 4, 5, 6}
            \draw[line width=0.1, color=gray!50, opacity=0.8] (\x, -2) -- (\x, 6);
        \foreach \y in {-2, -1, 0, 1, 2, 3, 4, 5, 6}
            \draw[line width=0.1, color=gray!50, opacity=0.8] (-1, \y) -- (6, \y);
        
        % Labels in axes:
        \foreach \x in {1, 2, 3, 4, 5}
            \node[anchor=south] at (\x, -0.5) {\small{$\x$}};
        \foreach \y in {-1, 1, 2, 3, 4, 5}
            \node[anchor=east] at (-0.1, \y) {\small{$\y$}};

        % Functions:
        \draw[line width=1, color=blue] plot[domain=-0.5:3.5] (\x, 2*\x - 1) node[anchor=north west] {$2x - y = 1$};
        \draw[line width=1, color=red] plot[domain=-1:6] (\x, 5 - \x) node[anchor=north east] {$x + y = 5$};

        % Intersection dot at (2,3):
        \draw[line width=0.5mm, color=black, fill=black] (2, 3) circle (0.05);

        % Intersection dot label:
        \node[anchor=west] at (2, 3) {$\left(2, 3\right)$};

        % Intersection grid:
        \draw[line width=0.1, color=black, opacity=0.8, dashed] (0, 3) -- (2, 3);
        \draw[line width=0.1, color=black, opacity=0.8, dashed] (2, 0) -- (2, 3);
    \end{tikzpicture}
    \caption{Gráfica del sistema de ecuaciones lineales (\ref{eq:1.1.1}).}
    \label{fig:1.1.1}
\end{figure}

Ahora bien, existe una equivalencia entre representar un sistema de ecuaciones lineales en forma de gráfica y representarlo en forma de vectores. Para ello, considere el sistema de ecuaciones (\ref{eq:1.1.1}). Puede ser representado como: \begin{align}
    \underbrace{\begin{bmatrix}2 \\ 1\end{bmatrix}}_{\vec{v_1}} x + \underbrace{\begin{bmatrix}-1 \\ 1\end{bmatrix}}_{\vec{v_2}} y &= \underbrace{\begin{bmatrix}1 \\ 5\end{bmatrix}}_{\vec{b}}, \quad x,y \in \mathbb{R}
    \label{eq:1.1.2}
\end{align}

Dónde la gráfica de cada vector es la siguiente:

\begin{figure}[H]
    \centering
    \begin{tikzpicture}
        % Axes:
        \draw[line width=0.4, color=black, arrows=->] (-2, 0) -- (3, 0)
                                                      (0, -1) -- (0, 6);
        
        % Labels:
        \node[anchor=west] at (3, 0) {$x$};
        \node[anchor=south] at (0, 6) {$y$};

        % Grid:
        \foreach \x in {-2, -1, 0, 1, 2, 3}
            \draw[line width=0.1, color=gray!50, opacity=0.8] (\x, -1) -- (\x, 6);
        \foreach \y in {-1, 0, 1, 2, 3, 4, 5, 6}
            \draw[line width=0.1, color=gray!50, opacity=0.8] (-2, \y) -- (3, \y);
        
        % Labels in axes:
        \foreach \x in {-1, -1, 1, 2}
            \node[anchor=south] at (\x, -0.5) {\small{$\x$}};
        \foreach \y in {1, 2, 3, 4, 5}
            \node[anchor=east] at (-0.1, \y) {\small{$\y$}};

        % Dependient vectors
        % \draw[line width=0.5mm, color=gray!80, dashed, opacity=0.5] (0, 0) -- (-3, 3);
        % \draw[line width=0.5mm, color=gray!80, dashed, opacity=0.5] (0, 0) -- (3, 1.5);
        
        % Vectors
        \draw[line width=0.5mm, color=red, arrows=->] (0, 0) -- (2, 1) node[anchor=south] {$\vec{v_1}$};
        \draw[line width=0.5mm, color=blue, arrows=->] (0, 0) -- (-1, 1) node[anchor=south] {$\vec{v_2}$};
        \draw[line width=0.5mm, color=orange, arrows=->] (0, 0) -- (1, 5) node[anchor=south] {$\vec{v_3}$};
    \end{tikzpicture}
    \caption{Óptica vectorial del sistema de ecuaciones (\ref{eq:1.1.1}).}
    \label{fig:1.1.2}
\end{figure}

\subsubsection{Álgebra de vectores}

\begin{enumerate}
    \item \textbf{Producto vectorial:} \begin{align*}
        v_1 \times v_2 = v_1 \times v_2^T &= \underbrace{\begin{pmatrix} \hspace*{0.1cm}\cdot\hspace*{0.1cm} \\ \hspace*{0.1cm}\cdot\hspace*{0.1cm} \end{pmatrix}}_{n\times1} \times \underbrace{\begin{pmatrix} \hspace*{0.1cm}\cdot\hspace*{0.1cm} & \hspace*{0.1cm}\cdot\hspace*{0.1cm} \end{pmatrix}}_{1\times n} = \underbrace{\begin{pmatrix} \hspace*{0.1cm}\cdot\hspace*{0.1cm} & \hspace*{0.1cm}\cdot\hspace*{0.1cm} \\ \hspace*{0.1cm}\cdot\hspace*{0.1cm} & \hspace*{0.1cm}\cdot\hspace*{0.1cm} \end{pmatrix}}_{\text{Matriz } n\times n}
    \end{align*}
    \item \textbf{Producto punto:} \begin{align*}
        v_1 \cdot v_2 &= v_1^T \cdot v_2 = \underbrace{\begin{pmatrix} \hspace*{0.1cm}\cdot\hspace*{0.1cm} & \hspace*{0.1cm}\cdot\hspace*{0.1cm} \end{pmatrix}}_{1\times n} \cdot \underbrace{\begin{pmatrix} \hspace*{0.1cm}\cdot\hspace*{0.1cm} \\ \hspace*{0.1cm}\cdot\hspace*{0.1cm} \end{pmatrix}}_{n\times1} = \underbrace{a}_{\mathbb{R}}
    \end{align*}
    \item \textbf{Producto por un escalar:} \begin{align*}
        \alpha v &= \alpha \begin{pmatrix} v_{11} \\ v_{21} \\ \vdots \\ v_{n1} \end{pmatrix} = \begin{pmatrix} \alpha v_{11} \\ \alpha v_{21} \\ \vdots \\ \alpha v_{n1} \end{pmatrix}
    \end{align*}
\end{enumerate}


\subsubsection{Variable compleja}
\label{sec:1_1}

Sea $z$ una variable compleja, la cual está definida como: \begin{align}
    z = a + ib,
    \label{eq:1.1.3}
\end{align}

dónde, $a$ y $b$ son números reales y $i^2 = -1$.

\paragraph*{Operaciones con números complejos:}

\begin{enumerate}
    \item Suma: \begin{align*}
        z_1 + z_2 &= a_1 + ib_1 + a_2 + ib_2 \\
                    &= (a_1 + a_2) + (ib_1 + ib_2) \\
                    &= (a_1 + a_2) + i(b_1 + b_2)
    \end{align*}
    \item Multiplicación: \begin{align*}
        z_1 z_2 &= (a_1 + ib_1)(a_2 + ib_2) \\
                &= (a_1a_2 - b_1b_2) + i(a_1b_2 + a_2b_1)
    \end{align*}
\end{enumerate}

\subsection{Vectores y matrices}

\begin{definition}[Vector renglón]
{
    \label{def:1.1.1}
    Sea $x$ un vector de $n$ dimensiones, entonces:
    \begin{align*}
        x = (x_1, x_2, \dots, x_n) \in \M_{(1 \times n)}
    \end{align*}
}
\end{definition}

\begin{definition}[Vector columna]
{
    \label{def:1.1.2}
    Sea $x$ un vector de $n$ dimensiones, entonces:
    \begin{align*}
        x = \begin{pmatrix}
            x_1 \\
            x_2 \\
            \vdots \\
            x_n
        \end{pmatrix} \in \M_{(n \times 1)}
    \end{align*}
}
\end{definition}

\begin{definition}[Matriz]
{
    \label{def:1.1.3}
    Sea $A$ una matriz de $m$ renglones y $n$ columnas, está definida como:
    \begin{align*}
        A = \begin{bmatrix}
            a_{11} & a_{12} & \cdots & a_{1n} \\
            a_{21} & a_{22} & \cdots & a_{2n} \\
            \vdots & \vdots & \ddots  & \vdots \\
            a_{m1} & a_{m2} & \cdots & a_{mn}
        \end{bmatrix} \in \M_{(m \times n)}
    \end{align*}
}
\end{definition}

\begin{definition}[Igualdad]
{
    \label{def:1.1.4}
    Sean $A$ y $B$ dos matrices de $m \times n$:
    \begin{align*}
        A = B \quad \Longleftrightarrow \quad \left[a_{ij}\right] = \left[b_{ij}\right], \quad \forall i \forall j
    \end{align*}
}
\end{definition}

\begin{definition}[Suma matricial]
{
    \label{def:1.1.5}
    Sean $A$, $B$ y $C$ matrices de $m \times n$:
    \begin{align*}
        A + B = C \Longleftrightarrow \left[ a_{ij} + b_{ij} \right] = \left[ c_{ij} \right]
    \end{align*}
}
\end{definition}

\begin{definition}[Multiplicación por un escalar]
{
    \label{def:1.1.6}
    Sean $A \in \M_{(m \times n)}$ y $\alpha \in \R$:
    \begin{align*}
        \alpha A = B, \quad B \in \M_{(m \times n)} \quad \Longleftrightarrow \quad \left[ \alpha a_{ij} \right] = \left[ b_{ij} \right]
    \end{align*}
}
\end{definition}

\begin{teorema}
{
    \label{thm:1}
    Sean $A$, $B$ y $C$ matrices de $m \times n$ y $\alpha, \beta$ números reales, entonces: \\

    \begin{enumerate}
        \item $A + \vec{0} = A$
        \item $0A = \vec{0}$
        \item $A + B = B + A$
        \item $(A + B) + C = A + ( B + C )$
        \item $\alpha ( A + B ) = \alpha A + \alpha B$
        \item $1A = A$
        \item $(\alpha + \beta) A = \alpha A + \beta A $
    \end{enumerate}
}
\end{teorema}

\subsection{Producto vectorial y producto matricial}

\begin{align*}
    \text{Sean } a = \begin{pmatrix}
        a_1 \\
        a_2 \\
        \vdots \\
        a_n
    \end{pmatrix}$ y $b = \begin{pmatrix}
        b_1 \\
        b_2 \\
        \vdots \\
        b_n
    \end{pmatrix}
\end{align*}

\begin{definition}[Producto punto]
{
    aaaaaa
}
\end{definition}
\pagebreak

% ####################################################################

\section{Espacios vectoriales y ecuaciones lineales}
\label{sec:espacios_vectoriales_y_ecuaciones_lineales}

\input{2. Espacios Vectoriales}
\pagebreak

\end{document}